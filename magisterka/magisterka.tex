\documentclass{pracamgr}
\usepackage{fontspec}
\usepackage{microtype}
\usepackage{polski}
\setmainfont{DejaVu Serif}
% Dane magistranta:

\author{Robert Blarbarucki}

\nralbumu{666999}

\title{Corthus - korpus równoległy z wyszukiwarką fonetyczną}

\tytulang{Corthus - a parallel corpus with a phonetic search engine}

\kierunek{Informatyka}

\opiekun{dra hab. Adama Przepiórkowskiego\\
  Instytut Informatyki }

\date{Wrzesień 2012}

%Podać dziedzinę wg klasyfikacji Socrates-Erasmus:
\dziedzina{11.3 Informatyka}

%Klasyfikacja tematyczna wedlug ACM
\klasyfikacja{
   H. Information Systems \\
   H.3 Information Storage and Retreival \\
   H.3.1 Content Analysis and Indexing }

% Słowa kluczowe:
\keywords{tłumaczenie, alignment, teksty równoległe, wyszukiwanie fonetyczne,
  metaphone}

\begin{document}
\maketitle

\begin{abstract}
  TODO
\end{abstract}

\tableofcontents
%\listoffigures
%\listoftables

\chapter*{Wprowadzenie}
\addcontentsline{toc}{chapter}{Wprowadzenie}

Cześć ludzie

привет мир

Γεια σας κόσμο


\chapter{Podstawowe pojęcia}\label{r:pojecia}


\begin{thebibliography}{99}
\addcontentsline{toc}{chapter}{Bibliografia}

\bibitem{orthlib} Teksty cerkiewnosłowiańskie:\\
  {\tt http://orthlib.ru/worship/}

\bibitem{analogion} Teksty starogreckie:\\
  {\tt http://analogion.gr/glt/}

\bibitem{liturgia} Teksty polskie:\\
  {\tt http://www.liturgia.cerkiew.pl/page.php?id=14}

\bibitem{hip} Opis standardu HIP (w jęz. rosyjskim):\\
  {\tt http://orthlib.ru/hip/hip-9.html}

\bibitem{irmologion} Czcionki z ligaturami potrzebne do wyświetlania
  tekstów w jęz. cerkiewnosłowiańskim:\\
  {\tt http://www.irmologion.ru/fonts.html}

\bibitem{unicode} Ирмологий, \textit{Unicode и церковно-славянские
  надстрочники. Что уже есть и что необходимо добавить}
  {\tt http://www.irmologion.ru/developer/fontdev4.html}

\bibitem{soundex} Algorytm Soundex:\\
  {\tt http://en.wikipedia.org/wiki/Soundex}

\bibitem{metaphone} Algorytm Metaphone:\\
  {\tt http://en.wikipedia.org/wiki/Metaphone},
  Hanging on the Metaphone, Lawrence Philips. Computer Language, Vol. 7, No. 12 (December), 1990.

\end{thebibliography}

\end{document}
